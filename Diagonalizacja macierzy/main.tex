\documentclass{article}
\usepackage{polski}
\usepackage[utf8]{inputenc}


\title{Diagonalizacja macierzy metodą potegową}
\author{Wiktoria Zaczyk}
\date{08.04.2021}

\usepackage{natbib}
\usepackage{graphicx}

\begin{document}

\maketitle

\section{Wstęp teoretyczny}
Na piątych laboratoriach zajęliśmy się diagonalizacją macierzy z pomocą metody potęgowej.

\newline\newline

\textbf{Wektor} to obiekt matematyczny opisywany za pomocą jego długości, zwrotu oraz kierunku. Notacja,
z którą wykorzystujemy w obliczeniach matematycznych do zapisu wektora to notacja macierzowa.
Trójelementowy wektor wierszowy:

\newline
\begin{center}
$v = [x~ y~ z]$
\end{center}

\newline\newline
\textbf{Iloczyn tensorowy wektorów}
\newline
W naszym przypadku będziemy korzystać z iloczynu tensorowego wektorów kolumnowego oraz wierszowego, więc w efekcie dostaniemy macierz o wymiarach $n \cdot n$, gdzie n to ilość elementów wektora.

\newline
\begin{center}
$v \otimes w = 
\left[\begin{array}{cc}
1\\
0
\end{array}\right]_v 
\otimes
~[1,0]_w =
\left[\begin{array}{cc}
1 \cdot [1,0]_w\\
0 \cdot [1,0]_w
\end{array}\right]
= \left[\begin{array}{cc}
1, 0\\
0,0
\end{array}\right]
$
\end{center}

\newline\newline
\textbf{Macierz} to tablica prostokątna, która zawiera liczby. Notacja w jakiej zapisujemy macierze widoczna
jest poniżej:

\newline
\begin{center}
$$
{A} =
\left[ \begin{array}{cc}
a_{11} & a_{12} \\
a_{21} & a_{22} \\
\end{array} \right]
$$

\end{center}
Na powyższym przykładzie widnieje macierz kwadratowa (ilość kolumn jest równa ilości wierszy) o
wymiarze 2. Wyróżniamy kilka rodzajów macierzy, poniżej te najważniejsze, które są istotne dla
przebiegu ćwiczenia.

\newline\newline
\textbf{Macierz diagonalna} to taka, która posiada wartości różne od zera jedynie na przekątnej (tzw. diagonali).
\newline\newline
\textbf{Macierz wstęgowa}  to taka, której wszystkie elementy są zerowe poza diagonalą i w jej pobliżu. Mając
daną macierz $n \cdot n$ jej elementy $a_i,j$ są niezerowe, gdy $i-k_1 \leq j \leq i+k_2$ gdzie $k_1_,_2 \geq 0$ określają
szerokość wstęgi.
\newline\newline
\textbf{Macierz trójdiagonalna}  to taka, która posiada wartości różne od zera jedynie na diagonali, oraz
pierwszej naddiagonali i pierwszej poddiagonali.

\section{Problem}
Wyznaczanie wartości własnych iteracyjnie przy użyciu metody potęgowej, opisanej algorytmem:

\begin{flushleft}

$W_0=A$ 
\hspace{4 cm}(inicjalizacja macierzy iterującej)\\
for($k = 0; k < K_v_a_l; k ++$)\\
\{\\
\hspace{2 mm} 
$x_k^0 = [1,1,...,1]$ 
\hspace{2.5 cm}
(inicjalizacja wektora startowego)\\
\hspace{2 mm}
for($i = 1; i \leq IT MAX; i ++$)\\
\hspace{2 mm}\{\\

\hspace{4 mm}
$x_k^i^+^1 = W_k x_k^i$\\

\hspace{4 mm}
$\lambda_k^i= \frac{(x_k^i^+^1)^T x_k^i}{(x_k^i)^T x_k^i}$\\

\hspace{4 mm}
$x_k^i= 
\frac{x_k^i^+^1}{||x_k^i^+^1||_2}$\\

\hspace{2 mm}\}\\

\hspace{2 mm}
$W_k_+_1 = W_k- 
\lambda_k ~x_k^i ~(x_k^i)^T$
\hspace{0.8 cm}
(iloczyn tensorowy)\\
\}\\
\end{flushleft}
gdzie:
\begin{itemize}
    \item k - numer wyznaczanej wartości własnej,
    \item i - numer iteracji dla określonego k,
    \item A - macierz pierwotna o wymiarze n, wypełniona następująco:
    
    \begin{center}
        $A_i_j = \frac{1+|i+j|}{1+|i-j|}$\\
   gdzie: i, j = 0, 1, . . . , n-1
    \end{center}
    
    \item $W_k$ - macierz iteracji,
    \item $\lambda_k^i$- przybliżenie k-tej wartości własnej w i-tej iteracji,
    \item $x_k^i$ - i-te przybliżenie k-tego wektora własnego,
    \item $K_v_a_l=n$ - liczba wartości własnych do wyznaczenia,
    \item IT MAX = 12  - maksymalna liczba iteracji dla każdego k,
\end{itemize}

\section{Cel zadania}
Za zadanie mieliśmy utworzyć macierz A rzędu $n=7$ i wyznaczyć jej wartości własne $\lambda_k^i$, które zapisaliśmy dla każdego k   oraz zachowaliśmy wyznaczone wektory własne w postaci kolumn macierzy X.

\begin{center}
    $X=[x_0,x_1,...,x_n_-_1]$
\end{center}
Co pozwoliło na wyznaczenie macierzy D:

\begin{center}
    $D=X^T\cdot A \cdot X$
\end{center}

\section{Wyniki}

\begin{flushleft}
Na poniższym rysunku możemy odczytać znalezione przybliżenia wartości własnych. Na osi X
znajduje się numer wartości własnej.
\end{flushleft}

\begin{figure}[h!]
\centering
\includegraphics[width=8cm]k
\caption{Wykres przybliżonych wartości własnych}
\label{fig:obrazek k}
\end{figure}

\begin{flushleft}
\textbf{Macierz diagonalna D}\\
W wyniku naszych działań otrzymaliśmy macierz diagonalną D następującej postaci:
\end{flushleft}
\begin{table}[h!]
    \centering
    \begin{tabular}{[c c c c c c ]}
    3.59586   & -8.896e-15 &   -2.289e-16   & -6.939e-17  &  -2.715e-16 &    2.706e-16  &  -5.551e-17 \\
   -8.993e-15   &     0.284988  &  -1.883e-06  &   -9.658e-09  &  -1.579e-09  &  -4.857e-17 &   -8.673e-18 \\
   -1.665e-16 &   -1.883e-06    &    0.122798  &   -0.002  &  -0.0001  &  -2.504e-12 &   -7.459e-17 \\
   -5.551e-16  &   -9.658e-09   &  -0.002   &     0.590378  &  -1.137e-08   & -1.388e-17 &   -9.194e-17 \\
   -3.331e-16  &  -1.58e-09  &  -0.0001  &  -1.137e-08   &    0.0865952  &  -1.608e-09   & -7.261e-15 \\
    1.110e-16  &   3.469e-17 &   -2.504e-12  &   2.776e-17  &  -1.608e-09    &    0.170974  &  -4.069e-09 \\
   -2.776e-16 &   -6.939e-18  &  -2.9490e-17  &  -5.55e-17 &   -7.298e-15 &   -4.069e-09   &    0.0981544    \\
    \end{tabular}

    \label{tab:my_label}
\end{table}
\section{Wnioski}
\begin{flushleft}
Na podstawie wyznaczania własności własnych można zauważyć, iż w większości przypadków
wystarczy około 6-7 iteracji, aby wartości własne już od tej pory były stałe. Można zauważyć, że macierz D jest macierzą prawie diagonalną,
elementy poza przekątną są bliskie zeru, jednak nie są mu równe ze względu na niedokładność obliczeń numerycznych. Mimo wszystko wartości są zadowalające, co w połączeniu z dobrą zbieżnością wartości własnych świadczy o skuteczności i prostocie wybranej metody.
\end{flushleft}

\end{document}
